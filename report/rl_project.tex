\documentclass{article}

% if you need to pass options to natbib, use, e.g.:
% \PassOptionsToPackage{numbers, compress}{natbib}
% before loading rl_project.

% to compile a camera-ready version, add the [final] option, e.g.:
 \usepackage[final]{rl_project}

% to avoid loading the natbib package, add option nonatbib:
% \usepackage[nonatbib]{rl_project}

\usepackage[utf8]{inputenc} % allow utf-8 input
\usepackage[T1]{fontenc}    % use 8-bit T1 fonts
\usepackage{hyperref}       % hyperlinks
\usepackage{url}            % simple URL typesetting
\usepackage{booktabs}       % professional-quality tables
\usepackage{amsfonts}       % blackboard math symbols
\usepackage{nicefrac}       % compact symbols for 1/2, etc.
\usepackage{microtype}      % microtypography
\usepackage{graphicx}


% Give your project report an appropriate title!

\title{RL Project}


% The \author macro works with any number of authors. There are two
% commands used to separate the names and addresses of multiple
% authors: \And and \AND.
%
% Using \And between authors leaves it to LaTeX to determine where to
% break the lines. Using \AND forces a line break at that point. So,
% if LaTeX puts 3 of 4 authors names on the first line, and the last
% on the second line, try using \AND instead of \And before the third
% author name.

\author{
  David S.~Hippocampus
  \\
  Department of Computer Science\\
  University of Bath\\
  Bath, BA2 7AY \\
  \texttt{hippo@bath.ac.uk} \\
  %% examples of more authors
  %% \And
  %% Coauthor \\
  %% Affiliation \\
  %% Address \\
  %% \texttt{email} \\
  %% \AND
  %% Coauthor \\
  %% Affiliation \\
  %% Address \\
  %% \texttt{email} \\
  %% \And
  %% Coauthor \\
  %% Affiliation \\
  %% Address \\
  %% \texttt{email} \\
  %% \And
  %% Coauthor \\
  %% Affiliation \\
  %% Address \\
  %% \texttt{email} \\
}

\begin{document}

\maketitle

\section{Problem Definition}

A clear, precise and concise description of your chosen problem, including the states, actions, transition dynamics, and the reward function. You will lose marks for an unclear, incorrect, or incomplete problem definition.

\section{Background}

A discussion of reinforcement learning methods that may be effective at solving your chosen problem, their strengths and weaknesses for your chosen problem, and any existing results in the scientific literature (or publicly available online) on your chosen problem or similar problems.

\subsection{DQN (Mirco)}

\subsection{A2C (Chris)}

\subsection{Async Q-Learning (Peter)}


\section{Method}
A description of the method(s) used to solve your chosen problem, an explanation of how these methods work (in your own words), and an explanation of why you chose these specific methods.
\subsection{DQN (Mirco)}

\subsection{A2C (Chris)}

\subsection{Async Q-Learning (Peter)}



\section{Results}

result comparison between the 3 approaches.
how quickly each agent learns
how well do they perform is absolte terms
how do they compare  with respect to a human player

\subsection{DQN (Mirco)}
\subsection{A2C (Chris)}
\subsection{Async Q-Learning (Peter)}



\section{Discussion}
    
An evaluation of how well you solved your chosen problem.

\subsection{DQN (Mirco)}
\subsection{A2C (Chris)}
\subsection{Async Q-Learning (Peter)}


\section{Future Work}

What other techniques can be used to improve further the performances of the 3 agents.

\section{Personal Experience}

A discussion of your personal experience with the project, such as difficulties or pleasant surprises you encountered while completing it.

\section*{References}
\small

\normalsize
\newpage
\section*{Appendices}
If you have additional content that you would like to include in the appendices, please do so here.
There is no limit to the length of your appendices, but we are not obliged to read them in their entirety while marking. The main body of your report should contain all essential information, and content in the appendices should be clearly referenced where it's needed elsewhere.
\subsection*{Appendix A: Example Appendix 1}
\subsection*{Appendix B: Example Appendix 2}

\end{document}
